\chapter{Uvod}

Področje poslovne inteligence \angl{Business Intelligence} se
ukvarja z razvojem in uporabo računalniško podprtih tehnik za
pridobivanje in analizo poslovnih podatkov. S tehnikami poslovne
inteligence si lahko pomagamo pri analizi preteklih in tekočih
poslovnih dogodkov, ter pri napovedovanju dogodkov v prihodnosti na
podlagi obstoječih podatkov in stanj. Podjetja uporabljajo te pristope
pri načrtovanju marketinških akcij, ciljnem oglašanju, komunikaciji z
uporabniki, sestavljanju uporabniško-usmerjenih spletnih strani,
ocenjevanju strank in segmentaciji trga. V družabnih omrežjih ter
nasploh na internetu pa je uporaba teh tehnik danes skoraj
vseobsegajoča, in jih, večinoma nevede, uporabljamo pri prav vsakem
dostopu do internetnih podatkov. Pravilna implementacija teh tehnik,
strani uporabe pravilno izbranih matematičnih in statističnih metod
ter računalniških algoritmov, danes, na medmrežju, mnogokrat pomeni
razliko med uspešno in propadlo spletno stranjo, ter posledično med
danes in nekdaj uspešnim podjetjem.

Tehnike poslovne inteligence naj bi predvsem podprle poslovno
odločanje. Programska orodja, ki te tehnike implementirajo, lahko zato
uvrstimo med sisteme za podporo odločanja. Za razliko od intuitivnega,
{\em ad hoc} odločanja, tehnike poslovne inteligence podprejo
odločanje z modeli, ki so zgrajeni na podlagi konkretnih podatkov.

Pri predmetu si bomo ogledali predvsem tehnike in algoritme poslovne
inteligence, torej tisti del, kjer je uporaba računalniškega znanja
ključna. Odgovore na bolj mehka vprašanja o splošni koristnosti teh
postopkov, družboslovnih vidikih umeščanja teh sistemov v podjetja,
vprašanju varnosti podatkov, zasebnosti in etiki pa bomo prepustili o
teh vprašanjih bolj podučenim.

V zadnjih letih smo pričam izjemnim spremembam okolja, iz katerega
podjetja in organizacije črpajo podatke. Lokalne, interne
transakcijske baze podatkov in podatkovna skladišča smo lahko pričeli
dopolnjevati, ali pa jih v posameznih primerih popolnoma nadomestili z
javno dostopnimi podatki in podatki iz družabnih omrežij. Tudi načini
pridobivanja podatkov so se precej spremenili. Od vprašalnikov, ki so
na primer od uporabnikov zahtevala odgovore na posebej zastavljena
vprašanja ali od njih terjala ocenjevanje določenih segmentov
poslovanja se danes zatekamo k mehkejšim načinom pridobivanja
podatkov. Na primer, merimo uporabo in opazujemo vzorce sprehodov po
spletnih straneh, uporabnike opremimo s karticami zvestobe (ki niso
ničesar drugega kot način identifikacije uporabnikov za namene
poslovne analitike), ali pa njihove seje v brskalniki zaznamujemo s
piškotki. Odgovor poslovne inteligence na te spremembe je bil razvoj
novih tehnik, kot so priporočilni sistemi in analiza omrežij. V
predmetu bo zato poudarek tudi na teh vrstah pristopov.

Pri predmetu bomo pregledali nekatere tipične postopkov za podatkovno
analitiko, ki se lahko uporabljajo v poslovnih okoljih in na družabnih
omrežjih. Poseben pomembna nam bo pri tem vizualizacija podatkov in
razlaga v podatkih odkritih vzorcev. V poslovnih okoljih analitika
namreč ni sama sebi namen, ampak je predvsem namenjena
odločevalcem. Ti lahko njene rezultate uporabljajo le, če so podani na
preprost in razumljiv način. Če že ne zaradi drugega, nam s pomočjo
zanimivih grafičnimi tehnik mnogokrat lažje uspe pridobiti zanimanje
odločevalcev (to je, direktorjev), ki nam potem posvetijo več časa in
se zato morda bolje poglobijo v problem.

Metode poslovne inteligence iz podatkov, na ta ali drugačen način,
gradijo modele. Ti so lahko vidni, namenjeni sporočanju odkritih
vzorcev določevalcem, ki potem iz njih gradijo poslovne in marketinške
strategije. Lahko pa so skriti, na primer v priporočilnih sistemih,
kjer nas mnogokrat ne zanima, zakaj nam algoritem predlaga to ali oni
(samo, da je predlog smiseln in uporaben!). Modele lahko razvijemo
tudi ročno, tipično tako, da jih opišemo v določenem formalnem
jeziku. Na koncu jih v obeh primerih, torej pri ročni ali pa gradnji
iz podatkov, uporabljajo odločevalci. Ti k mizi prinašajo različna
znanja, ki so pogosto tudi v konfliktu. Z reševanje tovrstnih
problemov, to je, s tehnikami odločanja v skupini, se bomo seznanili
pri koncu predmeta.
